
% Default to the notebook output style

    


% Inherit from the specified cell style.




    
\documentclass[11pt]{article}

    
    
    \usepackage[T1]{fontenc}
    % Nicer default font (+ math font) than Computer Modern for most use cases
    \usepackage{mathpazo}

    % Basic figure setup, for now with no caption control since it's done
    % automatically by Pandoc (which extracts ![](path) syntax from Markdown).
    \usepackage{graphicx}
    % We will generate all images so they have a width \maxwidth. This means
    % that they will get their normal width if they fit onto the page, but
    % are scaled down if they would overflow the margins.
    \makeatletter
    \def\maxwidth{\ifdim\Gin@nat@width>\linewidth\linewidth
    \else\Gin@nat@width\fi}
    \makeatother
    \let\Oldincludegraphics\includegraphics
    % Set max figure width to be 80% of text width, for now hardcoded.
    \renewcommand{\includegraphics}[1]{\Oldincludegraphics[width=.8\maxwidth]{#1}}
    % Ensure that by default, figures have no caption (until we provide a
    % proper Figure object with a Caption API and a way to capture that
    % in the conversion process - todo).
    \usepackage{caption}
    \DeclareCaptionLabelFormat{nolabel}{}
    \captionsetup{labelformat=nolabel}

    \usepackage{adjustbox} % Used to constrain images to a maximum size 
    \usepackage{xcolor} % Allow colors to be defined
    \usepackage{enumerate} % Needed for markdown enumerations to work
    \usepackage{geometry} % Used to adjust the document margins
    \usepackage{amsmath} % Equations
    \usepackage{amssymb} % Equations
    \usepackage{textcomp} % defines textquotesingle
    % Hack from http://tex.stackexchange.com/a/47451/13684:
    \AtBeginDocument{%
        \def\PYZsq{\textquotesingle}% Upright quotes in Pygmentized code
    }
    \usepackage{upquote} % Upright quotes for verbatim code
    \usepackage{eurosym} % defines \euro
    \usepackage[mathletters]{ucs} % Extended unicode (utf-8) support
    \usepackage[utf8x]{inputenc} % Allow utf-8 characters in the tex document
    \usepackage{fancyvrb} % verbatim replacement that allows latex
    \usepackage{grffile} % extends the file name processing of package graphics 
                         % to support a larger range 
    % The hyperref package gives us a pdf with properly built
    % internal navigation ('pdf bookmarks' for the table of contents,
    % internal cross-reference links, web links for URLs, etc.)
    \usepackage{hyperref}
    \usepackage{longtable} % longtable support required by pandoc >1.10
    \usepackage{booktabs}  % table support for pandoc > 1.12.2
    \usepackage[inline]{enumitem} % IRkernel/repr support (it uses the enumerate* environment)
    \usepackage[normalem]{ulem} % ulem is needed to support strikethroughs (\sout)
                                % normalem makes italics be italics, not underlines
    

    
    
    % Colors for the hyperref package
    \definecolor{urlcolor}{rgb}{0,.145,.698}
    \definecolor{linkcolor}{rgb}{.71,0.21,0.01}
    \definecolor{citecolor}{rgb}{.12,.54,.11}

    % ANSI colors
    \definecolor{ansi-black}{HTML}{3E424D}
    \definecolor{ansi-black-intense}{HTML}{282C36}
    \definecolor{ansi-red}{HTML}{E75C58}
    \definecolor{ansi-red-intense}{HTML}{B22B31}
    \definecolor{ansi-green}{HTML}{00A250}
    \definecolor{ansi-green-intense}{HTML}{007427}
    \definecolor{ansi-yellow}{HTML}{DDB62B}
    \definecolor{ansi-yellow-intense}{HTML}{B27D12}
    \definecolor{ansi-blue}{HTML}{208FFB}
    \definecolor{ansi-blue-intense}{HTML}{0065CA}
    \definecolor{ansi-magenta}{HTML}{D160C4}
    \definecolor{ansi-magenta-intense}{HTML}{A03196}
    \definecolor{ansi-cyan}{HTML}{60C6C8}
    \definecolor{ansi-cyan-intense}{HTML}{258F8F}
    \definecolor{ansi-white}{HTML}{C5C1B4}
    \definecolor{ansi-white-intense}{HTML}{A1A6B2}

    % commands and environments needed by pandoc snippets
    % extracted from the output of `pandoc -s`
    \providecommand{\tightlist}{%
      \setlength{\itemsep}{0pt}\setlength{\parskip}{0pt}}
    \DefineVerbatimEnvironment{Highlighting}{Verbatim}{commandchars=\\\{\}}
    % Add ',fontsize=\small' for more characters per line
    \newenvironment{Shaded}{}{}
    \newcommand{\KeywordTok}[1]{\textcolor[rgb]{0.00,0.44,0.13}{\textbf{{#1}}}}
    \newcommand{\DataTypeTok}[1]{\textcolor[rgb]{0.56,0.13,0.00}{{#1}}}
    \newcommand{\DecValTok}[1]{\textcolor[rgb]{0.25,0.63,0.44}{{#1}}}
    \newcommand{\BaseNTok}[1]{\textcolor[rgb]{0.25,0.63,0.44}{{#1}}}
    \newcommand{\FloatTok}[1]{\textcolor[rgb]{0.25,0.63,0.44}{{#1}}}
    \newcommand{\CharTok}[1]{\textcolor[rgb]{0.25,0.44,0.63}{{#1}}}
    \newcommand{\StringTok}[1]{\textcolor[rgb]{0.25,0.44,0.63}{{#1}}}
    \newcommand{\CommentTok}[1]{\textcolor[rgb]{0.38,0.63,0.69}{\textit{{#1}}}}
    \newcommand{\OtherTok}[1]{\textcolor[rgb]{0.00,0.44,0.13}{{#1}}}
    \newcommand{\AlertTok}[1]{\textcolor[rgb]{1.00,0.00,0.00}{\textbf{{#1}}}}
    \newcommand{\FunctionTok}[1]{\textcolor[rgb]{0.02,0.16,0.49}{{#1}}}
    \newcommand{\RegionMarkerTok}[1]{{#1}}
    \newcommand{\ErrorTok}[1]{\textcolor[rgb]{1.00,0.00,0.00}{\textbf{{#1}}}}
    \newcommand{\NormalTok}[1]{{#1}}
    
    % Additional commands for more recent versions of Pandoc
    \newcommand{\ConstantTok}[1]{\textcolor[rgb]{0.53,0.00,0.00}{{#1}}}
    \newcommand{\SpecialCharTok}[1]{\textcolor[rgb]{0.25,0.44,0.63}{{#1}}}
    \newcommand{\VerbatimStringTok}[1]{\textcolor[rgb]{0.25,0.44,0.63}{{#1}}}
    \newcommand{\SpecialStringTok}[1]{\textcolor[rgb]{0.73,0.40,0.53}{{#1}}}
    \newcommand{\ImportTok}[1]{{#1}}
    \newcommand{\DocumentationTok}[1]{\textcolor[rgb]{0.73,0.13,0.13}{\textit{{#1}}}}
    \newcommand{\AnnotationTok}[1]{\textcolor[rgb]{0.38,0.63,0.69}{\textbf{\textit{{#1}}}}}
    \newcommand{\CommentVarTok}[1]{\textcolor[rgb]{0.38,0.63,0.69}{\textbf{\textit{{#1}}}}}
    \newcommand{\VariableTok}[1]{\textcolor[rgb]{0.10,0.09,0.49}{{#1}}}
    \newcommand{\ControlFlowTok}[1]{\textcolor[rgb]{0.00,0.44,0.13}{\textbf{{#1}}}}
    \newcommand{\OperatorTok}[1]{\textcolor[rgb]{0.40,0.40,0.40}{{#1}}}
    \newcommand{\BuiltInTok}[1]{{#1}}
    \newcommand{\ExtensionTok}[1]{{#1}}
    \newcommand{\PreprocessorTok}[1]{\textcolor[rgb]{0.74,0.48,0.00}{{#1}}}
    \newcommand{\AttributeTok}[1]{\textcolor[rgb]{0.49,0.56,0.16}{{#1}}}
    \newcommand{\InformationTok}[1]{\textcolor[rgb]{0.38,0.63,0.69}{\textbf{\textit{{#1}}}}}
    \newcommand{\WarningTok}[1]{\textcolor[rgb]{0.38,0.63,0.69}{\textbf{\textit{{#1}}}}}
    
    
    % Define a nice break command that doesn't care if a line doesn't already
    % exist.
    \def\br{\hspace*{\fill} \\* }
    % Math Jax compatability definitions
    \def\gt{>}
    \def\lt{<}
    % Document parameters
    \title{linear\_regression\_project 3.5}
    
    
    

    % Pygments definitions
    
\makeatletter
\def\PY@reset{\let\PY@it=\relax \let\PY@bf=\relax%
    \let\PY@ul=\relax \let\PY@tc=\relax%
    \let\PY@bc=\relax \let\PY@ff=\relax}
\def\PY@tok#1{\csname PY@tok@#1\endcsname}
\def\PY@toks#1+{\ifx\relax#1\empty\else%
    \PY@tok{#1}\expandafter\PY@toks\fi}
\def\PY@do#1{\PY@bc{\PY@tc{\PY@ul{%
    \PY@it{\PY@bf{\PY@ff{#1}}}}}}}
\def\PY#1#2{\PY@reset\PY@toks#1+\relax+\PY@do{#2}}

\expandafter\def\csname PY@tok@w\endcsname{\def\PY@tc##1{\textcolor[rgb]{0.73,0.73,0.73}{##1}}}
\expandafter\def\csname PY@tok@c\endcsname{\let\PY@it=\textit\def\PY@tc##1{\textcolor[rgb]{0.25,0.50,0.50}{##1}}}
\expandafter\def\csname PY@tok@cp\endcsname{\def\PY@tc##1{\textcolor[rgb]{0.74,0.48,0.00}{##1}}}
\expandafter\def\csname PY@tok@k\endcsname{\let\PY@bf=\textbf\def\PY@tc##1{\textcolor[rgb]{0.00,0.50,0.00}{##1}}}
\expandafter\def\csname PY@tok@kp\endcsname{\def\PY@tc##1{\textcolor[rgb]{0.00,0.50,0.00}{##1}}}
\expandafter\def\csname PY@tok@kt\endcsname{\def\PY@tc##1{\textcolor[rgb]{0.69,0.00,0.25}{##1}}}
\expandafter\def\csname PY@tok@o\endcsname{\def\PY@tc##1{\textcolor[rgb]{0.40,0.40,0.40}{##1}}}
\expandafter\def\csname PY@tok@ow\endcsname{\let\PY@bf=\textbf\def\PY@tc##1{\textcolor[rgb]{0.67,0.13,1.00}{##1}}}
\expandafter\def\csname PY@tok@nb\endcsname{\def\PY@tc##1{\textcolor[rgb]{0.00,0.50,0.00}{##1}}}
\expandafter\def\csname PY@tok@nf\endcsname{\def\PY@tc##1{\textcolor[rgb]{0.00,0.00,1.00}{##1}}}
\expandafter\def\csname PY@tok@nc\endcsname{\let\PY@bf=\textbf\def\PY@tc##1{\textcolor[rgb]{0.00,0.00,1.00}{##1}}}
\expandafter\def\csname PY@tok@nn\endcsname{\let\PY@bf=\textbf\def\PY@tc##1{\textcolor[rgb]{0.00,0.00,1.00}{##1}}}
\expandafter\def\csname PY@tok@ne\endcsname{\let\PY@bf=\textbf\def\PY@tc##1{\textcolor[rgb]{0.82,0.25,0.23}{##1}}}
\expandafter\def\csname PY@tok@nv\endcsname{\def\PY@tc##1{\textcolor[rgb]{0.10,0.09,0.49}{##1}}}
\expandafter\def\csname PY@tok@no\endcsname{\def\PY@tc##1{\textcolor[rgb]{0.53,0.00,0.00}{##1}}}
\expandafter\def\csname PY@tok@nl\endcsname{\def\PY@tc##1{\textcolor[rgb]{0.63,0.63,0.00}{##1}}}
\expandafter\def\csname PY@tok@ni\endcsname{\let\PY@bf=\textbf\def\PY@tc##1{\textcolor[rgb]{0.60,0.60,0.60}{##1}}}
\expandafter\def\csname PY@tok@na\endcsname{\def\PY@tc##1{\textcolor[rgb]{0.49,0.56,0.16}{##1}}}
\expandafter\def\csname PY@tok@nt\endcsname{\let\PY@bf=\textbf\def\PY@tc##1{\textcolor[rgb]{0.00,0.50,0.00}{##1}}}
\expandafter\def\csname PY@tok@nd\endcsname{\def\PY@tc##1{\textcolor[rgb]{0.67,0.13,1.00}{##1}}}
\expandafter\def\csname PY@tok@s\endcsname{\def\PY@tc##1{\textcolor[rgb]{0.73,0.13,0.13}{##1}}}
\expandafter\def\csname PY@tok@sd\endcsname{\let\PY@it=\textit\def\PY@tc##1{\textcolor[rgb]{0.73,0.13,0.13}{##1}}}
\expandafter\def\csname PY@tok@si\endcsname{\let\PY@bf=\textbf\def\PY@tc##1{\textcolor[rgb]{0.73,0.40,0.53}{##1}}}
\expandafter\def\csname PY@tok@se\endcsname{\let\PY@bf=\textbf\def\PY@tc##1{\textcolor[rgb]{0.73,0.40,0.13}{##1}}}
\expandafter\def\csname PY@tok@sr\endcsname{\def\PY@tc##1{\textcolor[rgb]{0.73,0.40,0.53}{##1}}}
\expandafter\def\csname PY@tok@ss\endcsname{\def\PY@tc##1{\textcolor[rgb]{0.10,0.09,0.49}{##1}}}
\expandafter\def\csname PY@tok@sx\endcsname{\def\PY@tc##1{\textcolor[rgb]{0.00,0.50,0.00}{##1}}}
\expandafter\def\csname PY@tok@m\endcsname{\def\PY@tc##1{\textcolor[rgb]{0.40,0.40,0.40}{##1}}}
\expandafter\def\csname PY@tok@gh\endcsname{\let\PY@bf=\textbf\def\PY@tc##1{\textcolor[rgb]{0.00,0.00,0.50}{##1}}}
\expandafter\def\csname PY@tok@gu\endcsname{\let\PY@bf=\textbf\def\PY@tc##1{\textcolor[rgb]{0.50,0.00,0.50}{##1}}}
\expandafter\def\csname PY@tok@gd\endcsname{\def\PY@tc##1{\textcolor[rgb]{0.63,0.00,0.00}{##1}}}
\expandafter\def\csname PY@tok@gi\endcsname{\def\PY@tc##1{\textcolor[rgb]{0.00,0.63,0.00}{##1}}}
\expandafter\def\csname PY@tok@gr\endcsname{\def\PY@tc##1{\textcolor[rgb]{1.00,0.00,0.00}{##1}}}
\expandafter\def\csname PY@tok@ge\endcsname{\let\PY@it=\textit}
\expandafter\def\csname PY@tok@gs\endcsname{\let\PY@bf=\textbf}
\expandafter\def\csname PY@tok@gp\endcsname{\let\PY@bf=\textbf\def\PY@tc##1{\textcolor[rgb]{0.00,0.00,0.50}{##1}}}
\expandafter\def\csname PY@tok@go\endcsname{\def\PY@tc##1{\textcolor[rgb]{0.53,0.53,0.53}{##1}}}
\expandafter\def\csname PY@tok@gt\endcsname{\def\PY@tc##1{\textcolor[rgb]{0.00,0.27,0.87}{##1}}}
\expandafter\def\csname PY@tok@err\endcsname{\def\PY@bc##1{\setlength{\fboxsep}{0pt}\fcolorbox[rgb]{1.00,0.00,0.00}{1,1,1}{\strut ##1}}}
\expandafter\def\csname PY@tok@kc\endcsname{\let\PY@bf=\textbf\def\PY@tc##1{\textcolor[rgb]{0.00,0.50,0.00}{##1}}}
\expandafter\def\csname PY@tok@kd\endcsname{\let\PY@bf=\textbf\def\PY@tc##1{\textcolor[rgb]{0.00,0.50,0.00}{##1}}}
\expandafter\def\csname PY@tok@kn\endcsname{\let\PY@bf=\textbf\def\PY@tc##1{\textcolor[rgb]{0.00,0.50,0.00}{##1}}}
\expandafter\def\csname PY@tok@kr\endcsname{\let\PY@bf=\textbf\def\PY@tc##1{\textcolor[rgb]{0.00,0.50,0.00}{##1}}}
\expandafter\def\csname PY@tok@bp\endcsname{\def\PY@tc##1{\textcolor[rgb]{0.00,0.50,0.00}{##1}}}
\expandafter\def\csname PY@tok@fm\endcsname{\def\PY@tc##1{\textcolor[rgb]{0.00,0.00,1.00}{##1}}}
\expandafter\def\csname PY@tok@vc\endcsname{\def\PY@tc##1{\textcolor[rgb]{0.10,0.09,0.49}{##1}}}
\expandafter\def\csname PY@tok@vg\endcsname{\def\PY@tc##1{\textcolor[rgb]{0.10,0.09,0.49}{##1}}}
\expandafter\def\csname PY@tok@vi\endcsname{\def\PY@tc##1{\textcolor[rgb]{0.10,0.09,0.49}{##1}}}
\expandafter\def\csname PY@tok@vm\endcsname{\def\PY@tc##1{\textcolor[rgb]{0.10,0.09,0.49}{##1}}}
\expandafter\def\csname PY@tok@sa\endcsname{\def\PY@tc##1{\textcolor[rgb]{0.73,0.13,0.13}{##1}}}
\expandafter\def\csname PY@tok@sb\endcsname{\def\PY@tc##1{\textcolor[rgb]{0.73,0.13,0.13}{##1}}}
\expandafter\def\csname PY@tok@sc\endcsname{\def\PY@tc##1{\textcolor[rgb]{0.73,0.13,0.13}{##1}}}
\expandafter\def\csname PY@tok@dl\endcsname{\def\PY@tc##1{\textcolor[rgb]{0.73,0.13,0.13}{##1}}}
\expandafter\def\csname PY@tok@s2\endcsname{\def\PY@tc##1{\textcolor[rgb]{0.73,0.13,0.13}{##1}}}
\expandafter\def\csname PY@tok@sh\endcsname{\def\PY@tc##1{\textcolor[rgb]{0.73,0.13,0.13}{##1}}}
\expandafter\def\csname PY@tok@s1\endcsname{\def\PY@tc##1{\textcolor[rgb]{0.73,0.13,0.13}{##1}}}
\expandafter\def\csname PY@tok@mb\endcsname{\def\PY@tc##1{\textcolor[rgb]{0.40,0.40,0.40}{##1}}}
\expandafter\def\csname PY@tok@mf\endcsname{\def\PY@tc##1{\textcolor[rgb]{0.40,0.40,0.40}{##1}}}
\expandafter\def\csname PY@tok@mh\endcsname{\def\PY@tc##1{\textcolor[rgb]{0.40,0.40,0.40}{##1}}}
\expandafter\def\csname PY@tok@mi\endcsname{\def\PY@tc##1{\textcolor[rgb]{0.40,0.40,0.40}{##1}}}
\expandafter\def\csname PY@tok@il\endcsname{\def\PY@tc##1{\textcolor[rgb]{0.40,0.40,0.40}{##1}}}
\expandafter\def\csname PY@tok@mo\endcsname{\def\PY@tc##1{\textcolor[rgb]{0.40,0.40,0.40}{##1}}}
\expandafter\def\csname PY@tok@ch\endcsname{\let\PY@it=\textit\def\PY@tc##1{\textcolor[rgb]{0.25,0.50,0.50}{##1}}}
\expandafter\def\csname PY@tok@cm\endcsname{\let\PY@it=\textit\def\PY@tc##1{\textcolor[rgb]{0.25,0.50,0.50}{##1}}}
\expandafter\def\csname PY@tok@cpf\endcsname{\let\PY@it=\textit\def\PY@tc##1{\textcolor[rgb]{0.25,0.50,0.50}{##1}}}
\expandafter\def\csname PY@tok@c1\endcsname{\let\PY@it=\textit\def\PY@tc##1{\textcolor[rgb]{0.25,0.50,0.50}{##1}}}
\expandafter\def\csname PY@tok@cs\endcsname{\let\PY@it=\textit\def\PY@tc##1{\textcolor[rgb]{0.25,0.50,0.50}{##1}}}

\def\PYZbs{\char`\\}
\def\PYZus{\char`\_}
\def\PYZob{\char`\{}
\def\PYZcb{\char`\}}
\def\PYZca{\char`\^}
\def\PYZam{\char`\&}
\def\PYZlt{\char`\<}
\def\PYZgt{\char`\>}
\def\PYZsh{\char`\#}
\def\PYZpc{\char`\%}
\def\PYZdl{\char`\$}
\def\PYZhy{\char`\-}
\def\PYZsq{\char`\'}
\def\PYZdq{\char`\"}
\def\PYZti{\char`\~}
% for compatibility with earlier versions
\def\PYZat{@}
\def\PYZlb{[}
\def\PYZrb{]}
\makeatother


    % Exact colors from NB
    \definecolor{incolor}{rgb}{0.0, 0.0, 0.5}
    \definecolor{outcolor}{rgb}{0.545, 0.0, 0.0}



    
    % Prevent overflowing lines due to hard-to-break entities
    \sloppy 
    % Setup hyperref package
    \hypersetup{
      breaklinks=true,  % so long urls are correctly broken across lines
      colorlinks=true,
      urlcolor=urlcolor,
      linkcolor=linkcolor,
      citecolor=citecolor,
      }
    % Slightly bigger margins than the latex defaults
    
    \geometry{verbose,tmargin=1in,bmargin=1in,lmargin=1in,rmargin=1in}
    
    

    \begin{document}
    
    
    \maketitle
    
    

    
    \section{欢迎来到线性回归项目}\label{ux6b22ux8fceux6765ux5230ux7ebfux6027ux56deux5f52ux9879ux76ee}

若项目中的题目有困难没完成也没关系,我们鼓励你带着问题提交项目,评审人会给予你诸多帮助。

所有选做题都可以不做,不影响项目通过。如果你做了,那么项目评审会帮你批改,也会因为选做部分做错而判定为不通过。

其中非代码题可以提交手写后扫描的 pdf 文件,或使用 Latex
在文档中直接回答。

    \subsubsection{目录:}\label{ux76eeux5f55}

Section \ref{1-矩阵运算}\\
Section \ref{2-gaussian-jordan-消元法}\\
Section \ref{3-线性回归}

    \begin{Verbatim}[commandchars=\\\{\}]
{\color{incolor}In [{\color{incolor}1}]:} \PY{c+c1}{\PYZsh{} 任意选一个你喜欢的整数,这能帮你得到稳定的结果}
        \PY{n}{seed} \PY{o}{=} \PY{l+m+mi}{9999} \PY{c+c1}{\PYZsh{} TODO}
\end{Verbatim}


    \section{1 矩阵运算}\label{ux77e9ux9635ux8fd0ux7b97}

\subsection{1.1 创建一个 4*4
的单位矩阵}\label{ux521bux5efaux4e00ux4e2a-44-ux7684ux5355ux4f4dux77e9ux9635}

    \begin{Verbatim}[commandchars=\\\{\}]
{\color{incolor}In [{\color{incolor}2}]:} \PY{c+c1}{\PYZsh{} 这个项目设计来帮你熟悉 python list 和线性代数}
        \PY{c+c1}{\PYZsh{} 你不能调用任何NumPy以及相关的科学计算库来完成作业}
        
        
        \PY{c+c1}{\PYZsh{} 本项目要求矩阵统一使用二维列表表示,如下:}
        \PY{n}{A} \PY{o}{=} \PY{p}{[}\PY{p}{[}\PY{l+m+mi}{1}\PY{p}{,}\PY{l+m+mi}{2}\PY{p}{,}\PY{l+m+mi}{3}\PY{p}{]}\PY{p}{,} 
             \PY{p}{[}\PY{l+m+mi}{2}\PY{p}{,}\PY{l+m+mi}{3}\PY{p}{,}\PY{l+m+mi}{3}\PY{p}{]}\PY{p}{,} 
             \PY{p}{[}\PY{l+m+mi}{1}\PY{p}{,}\PY{l+m+mi}{2}\PY{p}{,}\PY{l+m+mi}{5}\PY{p}{]}\PY{p}{]}
        
        \PY{n}{B} \PY{o}{=} \PY{p}{[}\PY{p}{[}\PY{l+m+mi}{1}\PY{p}{,}\PY{l+m+mi}{2}\PY{p}{,}\PY{l+m+mi}{3}\PY{p}{,}\PY{l+m+mi}{5}\PY{p}{]}\PY{p}{,} 
             \PY{p}{[}\PY{l+m+mi}{2}\PY{p}{,}\PY{l+m+mi}{3}\PY{p}{,}\PY{l+m+mi}{3}\PY{p}{,}\PY{l+m+mi}{5}\PY{p}{]}\PY{p}{,} 
             \PY{p}{[}\PY{l+m+mi}{1}\PY{p}{,}\PY{l+m+mi}{2}\PY{p}{,}\PY{l+m+mi}{5}\PY{p}{,}\PY{l+m+mi}{1}\PY{p}{]}\PY{p}{]}
        
        \PY{c+c1}{\PYZsh{} 向量也用二维列表表示}
        \PY{n}{C} \PY{o}{=} \PY{p}{[}\PY{p}{[}\PY{l+m+mi}{1}\PY{p}{]}\PY{p}{,}
             \PY{p}{[}\PY{l+m+mi}{2}\PY{p}{]}\PY{p}{,}
             \PY{p}{[}\PY{l+m+mi}{3}\PY{p}{]}\PY{p}{]}
        
        \PY{c+c1}{\PYZsh{}TODO 创建一个 4*4 单位矩阵}
        \PY{n}{I} \PY{o}{=} \PY{p}{[}\PY{p}{[}\PY{l+m+mi}{1}\PY{p}{,}\PY{l+m+mi}{0}\PY{p}{,}\PY{l+m+mi}{0}\PY{p}{,}\PY{l+m+mi}{0}\PY{p}{]}\PY{p}{,} 
             \PY{p}{[}\PY{l+m+mi}{0}\PY{p}{,}\PY{l+m+mi}{1}\PY{p}{,}\PY{l+m+mi}{0}\PY{p}{,}\PY{l+m+mi}{0}\PY{p}{]}\PY{p}{,}
             \PY{p}{[}\PY{l+m+mi}{0}\PY{p}{,}\PY{l+m+mi}{0}\PY{p}{,}\PY{l+m+mi}{1}\PY{p}{,}\PY{l+m+mi}{0}\PY{p}{]}\PY{p}{,}
             \PY{p}{[}\PY{l+m+mi}{0}\PY{p}{,}\PY{l+m+mi}{0}\PY{p}{,}\PY{l+m+mi}{0}\PY{p}{,}\PY{l+m+mi}{1}\PY{p}{]}\PY{p}{]}
\end{Verbatim}


    \subsection{1.2
返回矩阵的行数和列数}\label{ux8fd4ux56deux77e9ux9635ux7684ux884cux6570ux548cux5217ux6570}

    \begin{Verbatim}[commandchars=\\\{\}]
{\color{incolor}In [{\color{incolor}3}]:} \PY{c+c1}{\PYZsh{} TODO 返回矩阵的行数和列数}
        \PY{k}{def} \PY{n+nf}{shape}\PY{p}{(}\PY{n}{M}\PY{p}{)}\PY{p}{:}
            \PY{k}{return} \PY{n+nb}{len}\PY{p}{(}\PY{n}{M}\PY{p}{)}\PY{p}{,}\PY{n+nb}{len}\PY{p}{(}\PY{n}{M}\PY{p}{[}\PY{l+m+mi}{0}\PY{p}{]}\PY{p}{)}
\end{Verbatim}


    \begin{Verbatim}[commandchars=\\\{\}]
{\color{incolor}In [{\color{incolor}4}]:} \PY{c+c1}{\PYZsh{} 运行以下代码测试你的 shape 函数}
        \PY{o}{\PYZpc{}}\PY{k}{run} \PYZhy{}i \PYZhy{}e test.py LinearRegressionTestCase.test\PYZus{}shape
\end{Verbatim}


    \begin{Verbatim}[commandchars=\\\{\}]
.
----------------------------------------------------------------------
Ran 1 test in 0.014s

OK

    \end{Verbatim}

    \subsection{1.3
每个元素四舍五入到特定小数数位}\label{ux6bcfux4e2aux5143ux7d20ux56dbux820dux4e94ux5165ux5230ux7279ux5b9aux5c0fux6570ux6570ux4f4d}

    \begin{Verbatim}[commandchars=\\\{\}]
{\color{incolor}In [{\color{incolor}5}]:} \PY{c+c1}{\PYZsh{} TODO 每个元素四舍五入到特定小数数位}
        \PY{c+c1}{\PYZsh{} 直接修改参数矩阵,无返回值}
        \PY{k}{def} \PY{n+nf}{matxRound}\PY{p}{(}\PY{n}{M}\PY{p}{,} \PY{n}{decPts}\PY{o}{=}\PY{l+m+mi}{4}\PY{p}{)}\PY{p}{:}
            \PY{k}{for} \PY{n}{i} \PY{o+ow}{in} \PY{n+nb}{range}\PY{p}{(}\PY{n+nb}{len}\PY{p}{(}\PY{n}{M}\PY{p}{)}\PY{p}{)}\PY{p}{:}
                \PY{k}{for} \PY{n}{j} \PY{o+ow}{in} \PY{n+nb}{range}\PY{p}{(}\PY{n+nb}{len}\PY{p}{(}\PY{n}{M}\PY{p}{[}\PY{l+m+mi}{0}\PY{p}{]}\PY{p}{)}\PY{p}{)}\PY{p}{:}
                    \PY{n}{M}\PY{p}{[}\PY{n}{i}\PY{p}{]}\PY{p}{[}\PY{n}{j}\PY{p}{]} \PY{o}{=} \PY{n+nb}{round}\PY{p}{(}\PY{n}{M}\PY{p}{[}\PY{n}{i}\PY{p}{]}\PY{p}{[}\PY{n}{j}\PY{p}{]}\PY{p}{,}\PY{n}{decPts}\PY{p}{)}
\end{Verbatim}


    \begin{Verbatim}[commandchars=\\\{\}]
{\color{incolor}In [{\color{incolor}6}]:} \PY{c+c1}{\PYZsh{} 运行以下代码测试你的 matxRound 函数}
        \PY{o}{\PYZpc{}}\PY{k}{run} \PYZhy{}i \PYZhy{}e test.py LinearRegressionTestCase.test\PYZus{}matxRound
\end{Verbatim}


    \begin{Verbatim}[commandchars=\\\{\}]
.
----------------------------------------------------------------------
Ran 1 test in 0.012s

OK

    \end{Verbatim}

    \subsection{1.4
计算矩阵的转置}\label{ux8ba1ux7b97ux77e9ux9635ux7684ux8f6cux7f6e}

    \begin{Verbatim}[commandchars=\\\{\}]
{\color{incolor}In [{\color{incolor}7}]:} \PY{c+c1}{\PYZsh{} TODO 计算矩阵的转置}
        \PY{k}{def} \PY{n+nf}{transpose}\PY{p}{(}\PY{n}{M}\PY{p}{)}\PY{p}{:}
            \PY{n}{\PYZus{}}\PY{p}{,} \PY{n}{cols} \PY{o}{=} \PY{n}{shape}\PY{p}{(}\PY{n}{M}\PY{p}{)}
            \PY{n}{MT} \PY{o}{=} \PY{p}{[}\PY{p}{[}\PY{n}{row}\PY{p}{[}\PY{n}{col}\PY{p}{]} \PY{k}{for} \PY{n}{row} \PY{o+ow}{in} \PY{n}{M}\PY{p}{]} \PY{k}{for} \PY{n}{col} \PY{o+ow}{in} \PY{n+nb}{range}\PY{p}{(}\PY{n}{cols}\PY{p}{)}\PY{p}{]}
            \PY{k}{return} \PY{n}{MT}
\end{Verbatim}


    \begin{Verbatim}[commandchars=\\\{\}]
{\color{incolor}In [{\color{incolor}8}]:} \PY{c+c1}{\PYZsh{} 运行以下代码测试你的 transpose 函数}
        \PY{o}{\PYZpc{}}\PY{k}{run} \PYZhy{}i \PYZhy{}e test.py LinearRegressionTestCase.test\PYZus{}transpose
\end{Verbatim}


    \begin{Verbatim}[commandchars=\\\{\}]
.
----------------------------------------------------------------------
Ran 1 test in 0.012s

OK

    \end{Verbatim}

    \subsection{1.5 计算矩阵乘法
AB}\label{ux8ba1ux7b97ux77e9ux9635ux4e58ux6cd5-ab}

    \begin{Verbatim}[commandchars=\\\{\}]
{\color{incolor}In [{\color{incolor}9}]:} \PY{c+c1}{\PYZsh{} TODO 计算矩阵乘法 AB,如果无法相乘则raise ValueError}
        \PY{k}{def} \PY{n+nf}{matxMultiply}\PY{p}{(}\PY{n}{A}\PY{p}{,} \PY{n}{B}\PY{p}{)}\PY{p}{:}
            \PY{n}{multiply} \PY{o}{=} \PY{p}{[}\PY{p}{]}
            \PY{k}{if} \PY{n+nb}{len}\PY{p}{(}\PY{n}{A}\PY{p}{[}\PY{l+m+mi}{0}\PY{p}{]}\PY{p}{)} \PY{o}{!=} \PY{n+nb}{len}\PY{p}{(}\PY{n}{B}\PY{p}{)}\PY{p}{:}
                \PY{k}{raise} \PY{n+ne}{ValueError}
                 
            \PY{n}{result} \PY{o}{=} \PY{p}{[}\PY{n+nb}{list}\PY{p}{(}\PY{n}{row}\PY{p}{)} \PY{k}{for} \PY{n}{row} \PY{o+ow}{in} \PY{n+nb}{zip}\PY{p}{(}\PY{o}{*}\PY{n}{B}\PY{p}{)}\PY{p}{]} 
          
            \PY{k}{for} \PY{n}{Al} \PY{o+ow}{in} \PY{n+nb}{range}\PY{p}{(}\PY{n+nb}{len}\PY{p}{(}\PY{n}{A}\PY{p}{)}\PY{p}{)}\PY{p}{:}
                \PY{n}{row} \PY{o}{=}\PY{p}{[}\PY{p}{]}
                \PY{k}{for} \PY{n}{Bl} \PY{o+ow}{in} \PY{n+nb}{range}\PY{p}{(}\PY{n+nb}{len}\PY{p}{(}\PY{n}{result}\PY{p}{)}\PY{p}{)}\PY{p}{:}
                    \PY{n}{num} \PY{o}{=} \PY{l+m+mi}{0}
                    \PY{k}{for} \PY{n}{Br} \PY{o+ow}{in} \PY{n+nb}{range}\PY{p}{(}\PY{n+nb}{len}\PY{p}{(}\PY{n}{result}\PY{p}{[}\PY{l+m+mi}{0}\PY{p}{]}\PY{p}{)}\PY{p}{)}\PY{p}{:}
                        \PY{n}{num} \PY{o}{+}\PY{o}{=}  \PY{n}{A}\PY{p}{[}\PY{n}{Al}\PY{p}{]}\PY{p}{[}\PY{n}{Br}\PY{p}{]} \PY{o}{*} \PY{n}{result}\PY{p}{[}\PY{n}{Bl}\PY{p}{]}\PY{p}{[}\PY{n}{Br}\PY{p}{]}
                    \PY{n}{row}\PY{o}{.}\PY{n}{append}\PY{p}{(}\PY{n}{num}\PY{p}{)}
                \PY{n}{multiply}\PY{o}{.}\PY{n}{append}\PY{p}{(}\PY{n}{row}\PY{p}{)}
            \PY{k}{return} \PY{n}{multiply}
\end{Verbatim}


    \begin{Verbatim}[commandchars=\\\{\}]
{\color{incolor}In [{\color{incolor}10}]:} \PY{c+c1}{\PYZsh{} 运行以下代码测试你的 matxMultiply 函数}
         \PY{o}{\PYZpc{}}\PY{k}{run} \PYZhy{}i \PYZhy{}e test.py LinearRegressionTestCase.test\PYZus{}matxMultiply
\end{Verbatim}


    \begin{Verbatim}[commandchars=\\\{\}]
.
----------------------------------------------------------------------
Ran 1 test in 0.122s

OK

    \end{Verbatim}

    \begin{center}\rule{0.5\linewidth}{\linethickness}\end{center}

\section{2 Gaussian Jordan
消元法}\label{gaussian-jordan-ux6d88ux5143ux6cd5}

\subsection{2.1
构造增广矩阵}\label{ux6784ux9020ux589eux5e7fux77e9ux9635}

\$ A =

\begin{bmatrix}
    a_{11}    & a_{12} & ... & a_{1n}\\
    a_{21}    & a_{22} & ... & a_{2n}\\
    a_{31}    & a_{22} & ... & a_{3n}\\
    ...    & ... & ... & ...\\
    a_{n1}    & a_{n2} & ... & a_{nn}\\
\end{bmatrix}

, b =

\begin{bmatrix}
    b_{1}  \\
    b_{2}  \\
    b_{3}  \\
    ...    \\
    b_{n}  \\
\end{bmatrix}

\$

返回 \$ Ab =

\begin{bmatrix}
    a_{11}    & a_{12} & ... & a_{1n} & b_{1}\\
    a_{21}    & a_{22} & ... & a_{2n} & b_{2}\\
    a_{31}    & a_{22} & ... & a_{3n} & b_{3}\\
    ...    & ... & ... & ...& ...\\
    a_{n1}    & a_{n2} & ... & a_{nn} & b_{n} \end{bmatrix}

\$

    \begin{Verbatim}[commandchars=\\\{\}]
{\color{incolor}In [{\color{incolor}11}]:} \PY{c+c1}{\PYZsh{} TODO 构造增广矩阵,假设A,b行数相同}
         \PY{k}{def} \PY{n+nf}{augmentMatrix}\PY{p}{(}\PY{n}{A}\PY{p}{,} \PY{n}{b}\PY{p}{)}\PY{p}{:}
             \PY{k}{return} \PY{p}{[}\PY{n}{row\PYZus{}a} \PY{o}{+} \PY{n}{row\PYZus{}b} \PY{k}{for} \PY{n}{row\PYZus{}a}\PY{p}{,} \PY{n}{row\PYZus{}b} \PY{o+ow}{in} \PY{n+nb}{zip}\PY{p}{(}\PY{n}{A}\PY{p}{,} \PY{n}{b}\PY{p}{)}\PY{p}{]}
\end{Verbatim}


    \begin{Verbatim}[commandchars=\\\{\}]
{\color{incolor}In [{\color{incolor}12}]:} \PY{c+c1}{\PYZsh{} 运行以下代码测试你的 augmentMatrix 函数}
         \PY{o}{\PYZpc{}}\PY{k}{run} \PYZhy{}i \PYZhy{}e test.py LinearRegressionTestCase.test\PYZus{}augmentMatrix
\end{Verbatim}


    \begin{Verbatim}[commandchars=\\\{\}]
.
----------------------------------------------------------------------
Ran 1 test in 0.008s

OK

    \end{Verbatim}

    \subsection{2.2 初等行变换}\label{ux521dux7b49ux884cux53d8ux6362}

\begin{itemize}
\tightlist
\item
  交换两行
\item
  把某行乘以一个非零常数
\item
  把某行加上另一行的若干倍:
\end{itemize}

    \begin{Verbatim}[commandchars=\\\{\}]
{\color{incolor}In [{\color{incolor}13}]:} \PY{c+c1}{\PYZsh{} TODO r1 \PYZlt{}\PYZhy{}\PYZhy{}\PYZhy{}\PYZgt{} r2}
         \PY{c+c1}{\PYZsh{} 直接修改参数矩阵,无返回值}
         \PY{k}{def} \PY{n+nf}{swapRows}\PY{p}{(}\PY{n}{M}\PY{p}{,} \PY{n}{r1}\PY{p}{,} \PY{n}{r2}\PY{p}{)}\PY{p}{:}
             \PY{n}{M}\PY{p}{[}\PY{n}{r1}\PY{p}{]}\PY{p}{,}\PY{n}{M}\PY{p}{[}\PY{n}{r2}\PY{p}{]} \PY{o}{=} \PY{n}{M}\PY{p}{[}\PY{n}{r2}\PY{p}{]}\PY{p}{,}\PY{n}{M}\PY{p}{[}\PY{n}{r1}\PY{p}{]}
\end{Verbatim}


    \begin{Verbatim}[commandchars=\\\{\}]
{\color{incolor}In [{\color{incolor}14}]:} \PY{c+c1}{\PYZsh{} 运行以下代码测试你的 swapRows 函数}
         \PY{o}{\PYZpc{}}\PY{k}{run} \PYZhy{}i \PYZhy{}e test.py LinearRegressionTestCase.test\PYZus{}swapRows
\end{Verbatim}


    \begin{Verbatim}[commandchars=\\\{\}]
.
----------------------------------------------------------------------
Ran 1 test in 0.004s

OK

    \end{Verbatim}

    \begin{Verbatim}[commandchars=\\\{\}]
{\color{incolor}In [{\color{incolor}15}]:} \PY{c+c1}{\PYZsh{} TODO r1 \PYZlt{}\PYZhy{}\PYZhy{}\PYZhy{} r1 * scale}
         \PY{c+c1}{\PYZsh{} scale为0是非法输入,要求 raise ValueError}
         \PY{c+c1}{\PYZsh{} 直接修改参数矩阵,无返回值}
         \PY{k}{def} \PY{n+nf}{scaleRow}\PY{p}{(}\PY{n}{M}\PY{p}{,} \PY{n}{r}\PY{p}{,} \PY{n}{scale}\PY{p}{)}\PY{p}{:}
             \PY{k}{if} \PY{n}{scale} \PY{o}{==} \PY{l+m+mi}{0}\PY{p}{:}
                 \PY{k}{raise} \PY{n+ne}{ValueError}\PY{p}{(}\PY{l+s+s2}{\PYZdq{}}\PY{l+s+s2}{scale CANNOT be 0}\PY{l+s+s2}{\PYZdq{}}\PY{p}{)}
             \PY{n}{M}\PY{p}{[}\PY{n}{r}\PY{p}{]} \PY{o}{=} \PY{p}{[}\PY{n}{e} \PY{o}{*} \PY{n}{scale} \PY{k}{for} \PY{n}{e} \PY{o+ow}{in} \PY{n}{M}\PY{p}{[}\PY{n}{r}\PY{p}{]}\PY{p}{]}
\end{Verbatim}


    \begin{Verbatim}[commandchars=\\\{\}]
{\color{incolor}In [{\color{incolor}16}]:} \PY{c+c1}{\PYZsh{} 运行以下代码测试你的 scaleRow 函数}
         \PY{o}{\PYZpc{}}\PY{k}{run} \PYZhy{}i \PYZhy{}e test.py LinearRegressionTestCase.test\PYZus{}scaleRow
\end{Verbatim}


    \begin{Verbatim}[commandchars=\\\{\}]
.
----------------------------------------------------------------------
Ran 1 test in 0.004s

OK

    \end{Verbatim}

    \begin{Verbatim}[commandchars=\\\{\}]
{\color{incolor}In [{\color{incolor}17}]:} \PY{c+c1}{\PYZsh{} TODO r1 \PYZlt{}\PYZhy{}\PYZhy{}\PYZhy{} r1 + r2*scale}
         \PY{c+c1}{\PYZsh{} 直接修改参数矩阵,无返回值}
         \PY{k}{def} \PY{n+nf}{addScaledRow}\PY{p}{(}\PY{n}{M}\PY{p}{,} \PY{n}{r1}\PY{p}{,} \PY{n}{r2}\PY{p}{,} \PY{n}{scale}\PY{p}{)}\PY{p}{:} 
             \PY{n}{M}\PY{p}{[}\PY{n}{r1}\PY{p}{]} \PY{o}{=} \PY{p}{[}\PY{n}{e1} \PY{o}{+} \PY{n}{e2} \PY{o}{*} \PY{n}{scale} \PY{k}{for} \PY{n}{e1}\PY{p}{,} \PY{n}{e2} \PY{o+ow}{in} \PY{n+nb}{zip}\PY{p}{(}\PY{n}{M}\PY{p}{[}\PY{n}{r1}\PY{p}{]}\PY{p}{,} \PY{n}{M}\PY{p}{[}\PY{n}{r2}\PY{p}{]}\PY{p}{)}\PY{p}{]}
\end{Verbatim}


    \begin{Verbatim}[commandchars=\\\{\}]
{\color{incolor}In [{\color{incolor}18}]:} \PY{c+c1}{\PYZsh{} 运行以下代码测试你的 addScaledRow 函数}
         \PY{o}{\PYZpc{}}\PY{k}{run} \PYZhy{}i \PYZhy{}e test.py LinearRegressionTestCase.test\PYZus{}addScaledRow
\end{Verbatim}


    \begin{Verbatim}[commandchars=\\\{\}]
.
----------------------------------------------------------------------
Ran 1 test in 0.004s

OK

    \end{Verbatim}

    \subsection{2.3 Gaussian Jordan 消元法求解 Ax =
b}\label{gaussian-jordan-ux6d88ux5143ux6cd5ux6c42ux89e3-ax-b}

    \subsubsection{2.3.1 算法}\label{ux7b97ux6cd5}

步骤1 检查A,b是否行数相同

步骤2 构造增广矩阵Ab

步骤3 逐列转换Ab为化简行阶梯形矩阵
\href{https://zh.wikipedia.org/wiki/\%E9\%98\%B6\%E6\%A2\%AF\%E5\%BD\%A2\%E7\%9F\%A9\%E9\%98\%B5\#.E5.8C.96.E7.AE.80.E5.90.8E.E7.9A.84-.7Bzh-hans:.E8.A1.8C.3B_zh-hant:.E5.88.97.3B.7D-.E9.98.B6.E6.A2.AF.E5.BD.A2.E7.9F.A9.E9.98.B5}{中文维基链接}

\begin{verbatim}
对于Ab的每一列(最后一列除外)
    当前列为列c
    寻找列c中 对角线以及对角线以下所有元素(行 c~N)的绝对值的最大值
    如果绝对值最大值为0
        那么A为奇异矩阵,返回None (你可以在选做问题2.4中证明为什么这里A一定是奇异矩阵)
    否则
        使用第一个行变换,将绝对值最大值所在行交换到对角线元素所在行(行c) 
        使用第二个行变换,将列c的对角线元素缩放为1
        多次使用第三个行变换,将列c的其他元素消为0
        
\end{verbatim}

步骤4 返回Ab的最后一列

\textbf{注:}
我们并没有按照常规方法先把矩阵转化为行阶梯形矩阵,再转换为化简行阶梯形矩阵,而是一步到位。如果你熟悉常规方法的话,可以思考一下两者的等价性。

    \subsubsection{2.3.2 算法推演}\label{ux7b97ux6cd5ux63a8ux6f14}

为了充分了解Gaussian Jordan消元法的计算流程,请根据Gaussian
Jordan消元法,分别手动推演矩阵A为\textbf{\emph{可逆矩阵}},矩阵A为\textbf{\emph{奇异矩阵}}两种情况。

    \paragraph{推演示例}\label{ux63a8ux6f14ux793aux4f8b}

\(Ab = \begin{bmatrix}  -7 & 5 & -1 & 1\\  1 & -3 & -8 & 1\\  -10 & -2 & 9 & 1\end{bmatrix}\)

\$ -\/-\textgreater{} \$
\(\begin{bmatrix}  1 & \frac{1}{5} & -\frac{9}{10} & -\frac{1}{10}\\  0 & -\frac{16}{5} & -\frac{71}{10} & \frac{11}{10}\\  0 & \frac{32}{5} & -\frac{73}{10} & \frac{3}{10}\end{bmatrix}\)

\$ -\/-\textgreater{} \$
\(\begin{bmatrix}  1 & 0 & -\frac{43}{64} & -\frac{7}{64}\\  0 & 1 & -\frac{73}{64} & \frac{3}{64}\\  0 & 0 & -\frac{43}{4} & \frac{5}{4}\end{bmatrix}\)

\$ -\/-\textgreater{} \$
\(\begin{bmatrix}  1 & 0 & 0 & -\frac{3}{16}\\  0 & 1 & 0 & -\frac{59}{688}\\  0 & 0 & 1 & -\frac{5}{43}\end{bmatrix}\)

\paragraph{推演有以下要求:}\label{ux63a8ux6f14ux6709ux4ee5ux4e0bux8981ux6c42}

\begin{enumerate}
\def\labelenumi{\arabic{enumi}.}
\tightlist
\item
  展示每一列的消元结果, 比如3*3的矩阵, 需要写三步
\item
  用分数来表示
\item
  分数不能再约分
\item
  我们已经给出了latex的语法,你只要把零改成你要的数字(或分数)即可
\item
  可以用\href{http://www.math.odu.edu/~bogacki/cgi-bin/lat.cgi?c=sys}{这个页面}检查你的答案(注意只是答案,
  推演步骤两者算法不一致)
\end{enumerate}

\emph{你可以用python的
\href{https://docs.python.org/2/library/fractions.html}{fractions}
模块辅助你的约分}

    \paragraph{分数的输入方法}\label{ux5206ux6570ux7684ux8f93ux5165ux65b9ux6cd5}

(双击这个区域就能看到语法啦)

示例一: \(\frac{n}{m}\)

示例二: \(-\frac{a}{b}\)

    \paragraph{以下开始你的尝试吧!}\label{ux4ee5ux4e0bux5f00ux59cbux4f60ux7684ux5c1dux8bd5ux5427}

    \begin{Verbatim}[commandchars=\\\{\}]
{\color{incolor}In [{\color{incolor}19}]:} \PY{c+c1}{\PYZsh{} 不要修改这里!}
         \PY{k+kn}{from} \PY{n+nn}{helper} \PY{k}{import} \PY{o}{*}
         \PY{n}{A} \PY{o}{=} \PY{n}{generateMatrix}\PY{p}{(}\PY{l+m+mi}{3}\PY{p}{,}\PY{n}{seed}\PY{p}{,}\PY{n}{singular}\PY{o}{=}\PY{k+kc}{False}\PY{p}{)}
         \PY{n}{b} \PY{o}{=} \PY{n}{np}\PY{o}{.}\PY{n}{ones}\PY{p}{(}\PY{n}{shape}\PY{o}{=}\PY{p}{(}\PY{l+m+mi}{3}\PY{p}{,}\PY{l+m+mi}{1}\PY{p}{)}\PY{p}{,}\PY{n}{dtype}\PY{o}{=}\PY{n+nb}{int}\PY{p}{)} \PY{c+c1}{\PYZsh{} it doesn\PYZsq{}t matter}
         \PY{n}{Ab} \PY{o}{=} \PY{n}{augmentMatrix}\PY{p}{(}\PY{n}{A}\PY{o}{.}\PY{n}{tolist}\PY{p}{(}\PY{p}{)}\PY{p}{,}\PY{n}{b}\PY{o}{.}\PY{n}{tolist}\PY{p}{(}\PY{p}{)}\PY{p}{)} \PY{c+c1}{\PYZsh{} 请确保你的增广矩阵已经写好了}
         \PY{n}{printInMatrixFormat}\PY{p}{(}\PY{n}{Ab}\PY{p}{,}\PY{n}{padding}\PY{o}{=}\PY{l+m+mi}{3}\PY{p}{,}\PY{n}{truncating}\PY{o}{=}\PY{l+m+mi}{0}\PY{p}{)}
\end{Verbatim}


    \begin{Verbatim}[commandchars=\\\{\}]
  7,  5,  3 ||  1 
 -5, -4,  6 ||  1 
  2, -2, -9 ||  1 

    \end{Verbatim}

    请按照算法的步骤3,逐步推演\textbf{\emph{可逆矩阵}}的变换。

在下面列出每一次循环体执行之后的增广矩阵(注意使用Section \ref{分数的输入方法})

\$ Ab =

\begin{bmatrix}
    7 & 5 & 3 & 1 \\
    -5 & -4 & 6 & 1 \\
    2 & -2 & -9 & 1 \end{bmatrix}

\$

\$ -\/-\textgreater{}

\begin{bmatrix}
    1 & \frac{5}{7} & \frac{3}{7} & \frac{1}{7} \\
    0 & -\frac{3}{35} & \frac{57}{35} & \frac{12}{35} \\
    0 & \frac{12}{7} & \frac{69}{14} & \frac{5}{14}\end{bmatrix}

\$

\$ -\/-\textgreater{} \$
\(\begin{bmatrix}  1 & 0 & \frac{3}{7} & \frac{43}{75}\\  0 & 1 & \frac{57}{35} & -\frac{53}{75}\\  0 & 0 & \frac{69}{14} & \frac{13}{75}\end{bmatrix}\)

\$ -\/-\textgreater{} \$
\(\begin{bmatrix}  1 & 0 & 0 & \frac{43}{75}\\  0 & 1 & 0 & -\frac{53}{75}\\  0 & 0 & 1 & \frac{13}{75}\end{bmatrix}\)

\(...\)

    \begin{Verbatim}[commandchars=\\\{\}]
{\color{incolor}In [{\color{incolor}20}]:} \PY{c+c1}{\PYZsh{} 不要修改这里!}
         \PY{n}{A} \PY{o}{=} \PY{n}{generateMatrix}\PY{p}{(}\PY{l+m+mi}{3}\PY{p}{,}\PY{n}{seed}\PY{p}{,}\PY{n}{singular}\PY{o}{=}\PY{k+kc}{True}\PY{p}{)}
         \PY{n}{b} \PY{o}{=} \PY{n}{np}\PY{o}{.}\PY{n}{ones}\PY{p}{(}\PY{n}{shape}\PY{o}{=}\PY{p}{(}\PY{l+m+mi}{3}\PY{p}{,}\PY{l+m+mi}{1}\PY{p}{)}\PY{p}{,}\PY{n}{dtype}\PY{o}{=}\PY{n+nb}{int}\PY{p}{)}
         \PY{n}{Ab} \PY{o}{=} \PY{n}{augmentMatrix}\PY{p}{(}\PY{n}{A}\PY{o}{.}\PY{n}{tolist}\PY{p}{(}\PY{p}{)}\PY{p}{,}\PY{n}{b}\PY{o}{.}\PY{n}{tolist}\PY{p}{(}\PY{p}{)}\PY{p}{)} \PY{c+c1}{\PYZsh{} 请确保你的增广矩阵已经写好了}
         \PY{n}{printInMatrixFormat}\PY{p}{(}\PY{n}{Ab}\PY{p}{,}\PY{n}{padding}\PY{o}{=}\PY{l+m+mi}{3}\PY{p}{,}\PY{n}{truncating}\PY{o}{=}\PY{l+m+mi}{0}\PY{p}{)}
\end{Verbatim}


    \begin{Verbatim}[commandchars=\\\{\}]
 -1,  6, -8 ||  1 
-10, -5,  5 ||  1 
 -9,  2, -4 ||  1 

    \end{Verbatim}

    请按照算法的步骤3,逐步推演\textbf{\emph{奇异矩阵}}的变换。

在下面列出每一次循环体执行之后的增广矩阵(注意使用Section \ref{分数的输入方法})

\$ Ab =

\begin{bmatrix}
    -1 & 6 & -8 & 1 \\
    -10 & -5 & 5 & 1 \\
    -9 & 2 & -4 & 1 \end{bmatrix}

\$

\$ -\/-\textgreater{}

\begin{bmatrix}
    1 & -6 & 8 & 0 \\
    0 & \frac{1}{2} & -\frac{1}{2} & 0 \\
    0 & \frac{2}{9} & -\frac{4}{9} & 0 \end{bmatrix}

\$

\$ -\/-\textgreater{} \$
\(\begin{bmatrix}  1 & 0 & \frac{8}{13} & 0 \\  0 & 1 & -\frac{17}{13} & 0 \\  0 & 0 & 0 & 1 \end{bmatrix}\)

\(...\)

    \subsubsection{2.3.3 实现 Gaussian Jordan
消元法}\label{ux5b9eux73b0-gaussian-jordan-ux6d88ux5143ux6cd5}

    \begin{Verbatim}[commandchars=\\\{\}]
{\color{incolor}In [{\color{incolor}21}]:} \PY{c+c1}{\PYZsh{} TODO 实现 Gaussain Jordan 方法求解 Ax = b}
         
         \PY{l+s+sd}{\PYZdq{}\PYZdq{}\PYZdq{} Gaussian Jordan 方法求解 Ax = b.}
         \PY{l+s+sd}{    参数}
         \PY{l+s+sd}{        A: 方阵 }
         \PY{l+s+sd}{        b: 列向量}
         \PY{l+s+sd}{        decPts: 四舍五入位数,默认为4}
         \PY{l+s+sd}{        epsilon: 判读是否为0的阈值,默认 1.0e\PYZhy{}16}
         \PY{l+s+sd}{        }
         \PY{l+s+sd}{    返回列向量 x 使得 Ax = b }
         \PY{l+s+sd}{    返回None,如果 A,b 高度不同}
         \PY{l+s+sd}{    返回None,如果 A 为奇异矩阵}
         \PY{l+s+sd}{\PYZdq{}\PYZdq{}\PYZdq{}}
         \PY{k+kn}{from} \PY{n+nn}{decimal} \PY{k}{import} \PY{n}{Decimal}
         \PY{k+kn}{from} \PY{n+nn}{fractions} \PY{k}{import} \PY{n}{Fraction}
         
         \PY{k}{def} \PY{n+nf}{gj\PYZus{}Solve}\PY{p}{(}\PY{n}{A}\PY{p}{,} \PY{n}{b}\PY{p}{,} \PY{n}{decPts}\PY{o}{=}\PY{l+m+mi}{4}\PY{p}{,} \PY{n}{epsilon}\PY{o}{=}\PY{l+m+mf}{1.0e\PYZhy{}16}\PY{p}{)}\PY{p}{:}
         \PY{c+c1}{\PYZsh{} 检查A,b是否行数相同}
             \PY{k}{if} \PY{p}{(}\PY{n+nb}{len}\PY{p}{(}\PY{n}{A}\PY{p}{)} \PY{o}{!=} \PY{n+nb}{len}\PY{p}{(}\PY{n}{b}\PY{p}{)}\PY{p}{)}\PY{p}{:}
                 \PY{k}{return} \PY{k+kc}{None}
         
         \PY{c+c1}{\PYZsh{} 构造增广矩阵Ab}
             \PY{n}{Ab} \PY{o}{=} \PY{n}{augmentMatrix}\PY{p}{(}\PY{n}{A}\PY{p}{,} \PY{n}{b}\PY{p}{)}
         
         \PY{c+c1}{\PYZsh{} 逐列转换Ab为简化行阶梯形矩阵}
             \PY{n}{rows}\PY{p}{,} \PY{n}{cols} \PY{o}{=} \PY{n}{shape}\PY{p}{(}\PY{n}{Ab}\PY{p}{)}
             \PY{c+c1}{\PYZsh{} TODO}
             \PY{k}{for} \PY{n}{c} \PY{o+ow}{in} \PY{n+nb}{range}\PY{p}{(}\PY{n}{cols}\PY{o}{\PYZhy{}}\PY{l+m+mi}{1}\PY{p}{)}\PY{p}{:}
                 
                 \PY{c+c1}{\PYZsh{}转置后迭代更加方便}
                 \PY{n}{AbT} \PY{o}{=} \PY{n}{transpose}\PY{p}{(}\PY{n}{Ab}\PY{p}{)}
                 \PY{n}{col} \PY{o}{=} \PY{n}{AbT}\PY{p}{[}\PY{n}{c}\PY{p}{]}
                 
                 \PY{c+c1}{\PYZsh{}列c(转置后的行c)中,c\PYZti{}N行(转置后的c\PYZti{}N列),每一个元素的绝对值列表}
                 \PY{n}{abs\PYZus{}cn} \PY{o}{=} \PY{p}{[}\PY{n+nb}{abs}\PY{p}{(}\PY{n}{e}\PY{p}{)} \PY{k}{for} \PY{n}{e} \PY{o+ow}{in} \PY{n}{col}\PY{p}{[}\PY{n}{c}\PY{p}{:}\PY{p}{]}\PY{p}{]}
                 
                 \PY{c+c1}{\PYZsh{}绝对值列表的最大值}
                 \PY{n}{max\PYZus{}abs} \PY{o}{=} \PY{n+nb}{max}\PY{p}{(}\PY{n}{abs\PYZus{}cn}\PY{p}{)}
                 \PY{n}{max\PYZus{}index} \PY{o}{=} \PY{n}{abs\PYZus{}cn}\PY{o}{.}\PY{n}{index}\PY{p}{(}\PY{n}{max\PYZus{}abs}\PY{p}{)} \PY{o}{+} \PY{n}{c}
                 \PY{c+c1}{\PYZsh{}max\PYZus{}value = col[max\PYZus{}index]}
                 \PY{c+c1}{\PYZsh{}因为精度的问题,所以绝对值最大值小于 epsilon 就看作是等于 0}
                 \PY{k}{if} \PY{n}{max\PYZus{}abs} \PY{o}{\PYZlt{}}\PY{o}{=} \PY{n}{epsilon}\PY{p}{:}
                     \PY{k}{return} \PY{k+kc}{None}
         
                 \PY{c+c1}{\PYZsh{}最大值所在行(转置后的列)}
                 \PY{c+c1}{\PYZsh{}max\PYZus{}index = abs\PYZus{}cn.index(max\PYZus{}value) + c}
                 
                 \PY{c+c1}{\PYZsh{}使用第一个行变换,将绝对值最大值所在行交换到对角线元素所在行(行c)}
                 \PY{n}{swapRows}\PY{p}{(}\PY{n}{Ab}\PY{p}{,} \PY{n}{c}\PY{p}{,} \PY{n}{max\PYZus{}index}\PY{p}{)}
                 
                 \PY{c+c1}{\PYZsh{}使用第二个行变换,将列c的对角线元素缩放为1}
                 \PY{n}{scale\PYZus{}c} \PY{o}{=} \PY{l+m+mf}{1.0} \PY{o}{/} \PY{n}{Ab}\PY{p}{[}\PY{n}{c}\PY{p}{]}\PY{p}{[}\PY{n}{c}\PY{p}{]}
                 \PY{n}{scaleRow}\PY{p}{(}\PY{n}{Ab}\PY{p}{,} \PY{n}{c}\PY{p}{,} \PY{n}{scale\PYZus{}c}\PY{p}{)}
                 
                 \PY{c+c1}{\PYZsh{}多次使用第三个行变换,将列c的其他元素消为0}
                 \PY{k}{for} \PY{n}{i} \PY{o+ow}{in} \PY{n+nb}{range}\PY{p}{(}\PY{n+nb}{len}\PY{p}{(}\PY{n}{A}\PY{p}{)}\PY{p}{)}\PY{p}{:}
                     \PY{k}{if} \PY{n}{Ab}\PY{p}{[}\PY{n}{i}\PY{p}{]}\PY{p}{[}\PY{n}{c}\PY{p}{]} \PY{o}{!=} \PY{l+m+mi}{0} \PY{o+ow}{and} \PY{n}{i} \PY{o}{!=} \PY{n}{c}\PY{p}{:}
                         \PY{n}{addScaledRow}\PY{p}{(}\PY{n}{Ab}\PY{p}{,} \PY{n}{i}\PY{p}{,} \PY{n}{c}\PY{p}{,} \PY{o}{\PYZhy{}}\PY{n}{Ab}\PY{p}{[}\PY{n}{i}\PY{p}{]}\PY{p}{[}\PY{n}{c}\PY{p}{]}\PY{p}{)}
             
             \PY{n}{matxRound}\PY{p}{(}\PY{n}{Ab}\PY{p}{)}            
                     
         \PY{c+c1}{\PYZsh{} 返回Ab的最后一列}
             \PY{k}{return} \PY{p}{[}\PY{p}{[}\PY{n}{e\PYZus{}last}\PY{p}{]} \PY{k}{for} \PY{n}{e\PYZus{}last} \PY{o+ow}{in} \PY{n}{transpose}\PY{p}{(}\PY{n}{Ab}\PY{p}{)}\PY{p}{[}\PY{o}{\PYZhy{}}\PY{l+m+mi}{1}\PY{p}{]}\PY{p}{]}
\end{Verbatim}


    \begin{Verbatim}[commandchars=\\\{\}]
{\color{incolor}In [{\color{incolor}22}]:} \PY{c+c1}{\PYZsh{} TODO 实现 Gaussain Jordan 方法求解 Ax = b}
         
         \PY{l+s+sd}{\PYZdq{}\PYZdq{}\PYZdq{} Gaussian Jordan 方法求解 Ax = b.}
         \PY{l+s+sd}{    参数}
         \PY{l+s+sd}{        A: 方阵 }
         \PY{l+s+sd}{        b: 列向量}
         \PY{l+s+sd}{        decPts: 四舍五入位数,默认为4}
         \PY{l+s+sd}{        epsilon: 判读是否为0的阈值,默认 1.0e\PYZhy{}16}
         \PY{l+s+sd}{        }
         \PY{l+s+sd}{    返回列向量 x 使得 Ax = b }
         \PY{l+s+sd}{    返回None,如果 A,b 高度不同}
         \PY{l+s+sd}{    返回None,如果 A 为奇异矩阵}
         \PY{l+s+sd}{\PYZdq{}\PYZdq{}\PYZdq{}}
         
         \PY{k+kn}{from} \PY{n+nn}{decimal} \PY{k}{import} \PY{n}{Decimal}
         \PY{k+kn}{from} \PY{n+nn}{fractions} \PY{k}{import} \PY{n}{Fraction}
         
         \PY{k}{def} \PY{n+nf}{gj\PYZus{}Solve}\PY{p}{(}\PY{n}{A}\PY{p}{,} \PY{n}{b}\PY{p}{,} \PY{n}{decPts}\PY{o}{=}\PY{l+m+mi}{4}\PY{p}{,} \PY{n}{epsilon}\PY{o}{=}\PY{l+m+mf}{1.0e\PYZhy{}16}\PY{p}{)}\PY{p}{:}
         \PY{c+c1}{\PYZsh{} 检查A,b是否行数相同}
             
         
         \PY{c+c1}{\PYZsh{} 构造增广矩阵Ab}
             \PY{n}{Ab} \PY{o}{=} \PY{n}{augmentMatrix}\PY{p}{(}\PY{n}{A}\PY{p}{,} \PY{n}{b}\PY{p}{)}
         
         \PY{c+c1}{\PYZsh{} 逐列转换Ab为简化行阶梯形矩阵}
             \PY{n}{rows}\PY{p}{,} \PY{n}{cols} \PY{o}{=} \PY{n}{shape}\PY{p}{(}\PY{n}{Ab}\PY{p}{)}
             \PY{c+c1}{\PYZsh{} TODO}
             \PY{k}{for} \PY{n}{c} \PY{o+ow}{in} \PY{n+nb}{range}\PY{p}{(}\PY{n}{cols}\PY{o}{\PYZhy{}}\PY{l+m+mi}{1}\PY{p}{)}\PY{p}{:}
                 
                 \PY{c+c1}{\PYZsh{}转置后迭代更加方便}
                 \PY{n}{AbT} \PY{o}{=} \PY{n}{transpose}\PY{p}{(}\PY{n}{Ab}\PY{p}{)}
                 \PY{n}{col} \PY{o}{=} \PY{n}{AbT}\PY{p}{[}\PY{n}{c}\PY{p}{]}
                 
                 \PY{c+c1}{\PYZsh{}列c(转置后的行c)中,c\PYZti{}N行(转置后的c\PYZti{}N列),每一个元素的绝对值列表}
                 \PY{n}{abs\PYZus{}cn} \PY{o}{=} \PY{p}{[}\PY{n+nb}{abs}\PY{p}{(}\PY{n}{e}\PY{p}{)} \PY{k}{for} \PY{n}{e} \PY{o+ow}{in} \PY{n}{col}\PY{p}{[}\PY{n}{c}\PY{p}{:}\PY{p}{]}\PY{p}{]}
                 
                 \PY{c+c1}{\PYZsh{}绝对值列表的最大值}
                 \PY{n}{max\PYZus{}abs} \PY{o}{=} \PY{n+nb}{max}\PY{p}{(}\PY{n}{abs\PYZus{}cn}\PY{p}{)}
                 \PY{n}{max\PYZus{}index} \PY{o}{=} \PY{n}{abs\PYZus{}cn}\PY{o}{.}\PY{n}{index}\PY{p}{(}\PY{n}{max\PYZus{}abs}\PY{p}{)} \PY{o}{+} \PY{n}{c}
                 \PY{c+c1}{\PYZsh{}max\PYZus{}value = col[max\PYZus{}index]}
                 \PY{c+c1}{\PYZsh{}因为精度的问题,所以绝对值最大值小于 epsilon 就看作是等于 0}
                 \PY{k}{if} \PY{n}{max\PYZus{}abs} \PY{o}{\PYZlt{}}\PY{o}{=} \PY{n}{epsilon}\PY{p}{:}
                     \PY{k}{return} \PY{k+kc}{None}
         
                 \PY{c+c1}{\PYZsh{}最大值所在行(转置后的列)}
                 \PY{c+c1}{\PYZsh{}max\PYZus{}index = abs\PYZus{}cn.index(max\PYZus{}value) + c}
                 
                 \PY{c+c1}{\PYZsh{}使用第一个行变换,将绝对值最大值所在行交换到对角线元素所在行(行c)}
                 \PY{n}{swapRows}\PY{p}{(}\PY{n}{Ab}\PY{p}{,} \PY{n}{c}\PY{p}{,} \PY{n}{max\PYZus{}index}\PY{p}{)}
                 
                 \PY{c+c1}{\PYZsh{}使用第二个行变换,将列c的对角线元素缩放为1}
                 \PY{n}{scale\PYZus{}c} \PY{o}{=} \PY{l+m+mf}{1.0} \PY{o}{/} \PY{n}{Ab}\PY{p}{[}\PY{n}{c}\PY{p}{]}\PY{p}{[}\PY{n}{c}\PY{p}{]}
                 \PY{n}{scaleRow}\PY{p}{(}\PY{n}{Ab}\PY{p}{,} \PY{n}{c}\PY{p}{,} \PY{n}{scale\PYZus{}c}\PY{p}{)}
                 
                 \PY{c+c1}{\PYZsh{}多次使用第三个行变换,将列c的其他元素消为0}
                 \PY{k}{for} \PY{n}{i} \PY{o+ow}{in} \PY{n+nb}{range}\PY{p}{(}\PY{n+nb}{len}\PY{p}{(}\PY{n}{A}\PY{p}{)}\PY{p}{)}\PY{p}{:}
                     \PY{k}{if} \PY{n}{Ab}\PY{p}{[}\PY{n}{i}\PY{p}{]}\PY{p}{[}\PY{n}{c}\PY{p}{]} \PY{o}{!=} \PY{l+m+mi}{0} \PY{o+ow}{and} \PY{n}{i} \PY{o}{!=} \PY{n}{c}\PY{p}{:}
                         \PY{n}{addScaledRow}\PY{p}{(}\PY{n}{Ab}\PY{p}{,} \PY{n}{i}\PY{p}{,} \PY{n}{c}\PY{p}{,} \PY{o}{\PYZhy{}}\PY{n}{Ab}\PY{p}{[}\PY{n}{i}\PY{p}{]}\PY{p}{[}\PY{n}{c}\PY{p}{]}\PY{p}{)}
             
             \PY{n}{matxRound}\PY{p}{(}\PY{n}{Ab}\PY{p}{)}            
                     
         \PY{c+c1}{\PYZsh{} 返回Ab的最后一列}
             \PY{k}{return} \PY{p}{[}\PY{p}{[}\PY{n}{e\PYZus{}last}\PY{p}{]} \PY{k}{for} \PY{n}{e\PYZus{}last} \PY{o+ow}{in} \PY{n}{transpose}\PY{p}{(}\PY{n}{Ab}\PY{p}{)}\PY{p}{[}\PY{o}{\PYZhy{}}\PY{l+m+mi}{1}\PY{p}{]}\PY{p}{]}
\end{Verbatim}


    \begin{Verbatim}[commandchars=\\\{\}]
{\color{incolor}In [{\color{incolor}23}]:} \PY{c+c1}{\PYZsh{} 运行以下代码测试你的 gj\PYZus{}Solve 函数}
         \PY{o}{\PYZpc{}}\PY{k}{run} \PYZhy{}i \PYZhy{}e test.py LinearRegressionTestCase.test\PYZus{}gj\PYZus{}Solve
\end{Verbatim}


    \begin{Verbatim}[commandchars=\\\{\}]
.
----------------------------------------------------------------------
Ran 1 test in 4.030s

OK

    \end{Verbatim}

    \subsection{(选做) 2.4
算法正确判断了奇异矩阵:}\label{ux9009ux505a-2.4-ux7b97ux6cd5ux6b63ux786eux5224ux65adux4e86ux5947ux5f02ux77e9ux9635}

在算法的步骤3
中,如果发现某一列对角线和对角线以下所有元素都为0,那么则断定这个矩阵为奇异矩阵。

我们用正式的语言描述这个命题,并证明为真。

证明下面的命题:

\textbf{如果方阵 A 可以被分为4个部分: }

\$ A =

\begin{bmatrix}
    I    & X \\
    Z    & Y \\
\end{bmatrix}

, \text{其中 I 为单位矩阵,Z 为全0矩阵,Y 的第一列全0}\$,

\textbf{那么A为奇异矩阵。}

提示:从多种角度都可以完成证明 - 考虑矩阵 Y 和 矩阵 A 的秩 - 考虑矩阵 Y
和 矩阵 A 的行列式 - 考虑矩阵 A 的某一列是其他列的线性组合

    TODO 证明:

    \section{3 线性回归}\label{ux7ebfux6027ux56deux5f52}

    \subsection{3.1
随机生成样本点}\label{ux968fux673aux751fux6210ux6837ux672cux70b9}

    \begin{Verbatim}[commandchars=\\\{\}]
{\color{incolor}In [{\color{incolor}24}]:} \PY{c+c1}{\PYZsh{} 不要修改这里!}
         \PY{o}{\PYZpc{}}\PY{k}{matplotlib} notebook
         \PY{k+kn}{from} \PY{n+nn}{helper} \PY{k}{import} \PY{o}{*}
         
         \PY{n}{X}\PY{p}{,}\PY{n}{Y} \PY{o}{=} \PY{n}{generatePoints2D}\PY{p}{(}\PY{n}{seed}\PY{p}{)}
         \PY{n}{vs\PYZus{}scatter\PYZus{}2d}\PY{p}{(}\PY{n}{X}\PY{p}{,} \PY{n}{Y}\PY{p}{)}
\end{Verbatim}


    
    \begin{verbatim}
<IPython.core.display.Javascript object>
    \end{verbatim}

    
    
    \begin{verbatim}
<IPython.core.display.HTML object>
    \end{verbatim}

    
    \subsection{3.2
拟合一条直线}\label{ux62dfux5408ux4e00ux6761ux76f4ux7ebf}

\subsubsection{3.2.1
猜测一条直线}\label{ux731cux6d4bux4e00ux6761ux76f4ux7ebf}

    \begin{Verbatim}[commandchars=\\\{\}]
{\color{incolor}In [{\color{incolor}25}]:} \PY{c+c1}{\PYZsh{}TODO 请选择最适合的直线 y = mx + b}
         \PY{n}{m1} \PY{o}{=} \PY{l+m+mi}{3}
         \PY{n}{b1} \PY{o}{=} \PY{l+m+mi}{13}
         
         \PY{c+c1}{\PYZsh{} 不要修改这里!}
         \PY{n}{vs\PYZus{}scatter\PYZus{}2d}\PY{p}{(}\PY{n}{X}\PY{p}{,} \PY{n}{Y}\PY{p}{,} \PY{n}{m1}\PY{p}{,} \PY{n}{b1}\PY{p}{)}
\end{Verbatim}


    
    \begin{verbatim}
<IPython.core.display.Javascript object>
    \end{verbatim}

    
    
    \begin{verbatim}
<IPython.core.display.HTML object>
    \end{verbatim}

    
    \subsubsection{3.2.2 计算平均平方误差
(MSE)}\label{ux8ba1ux7b97ux5e73ux5747ux5e73ux65b9ux8befux5dee-mse}

    我们要编程计算所选直线的平均平方误差(MSE),
即数据集中每个点到直线的Y方向距离的平方的平均数,表达式如下: \[
MSE = \frac{1}{n}\sum_{i=1}^{n}{(y_i - mx_i - b)^2}
\]

    \begin{Verbatim}[commandchars=\\\{\}]
{\color{incolor}In [{\color{incolor}26}]:} \PY{c+c1}{\PYZsh{} TODO 实现以下函数并输出所选直线的MSE}
         \PY{k}{def} \PY{n+nf}{calculateMSE2D}\PY{p}{(}\PY{n}{X}\PY{p}{,}\PY{n}{Y}\PY{p}{,}\PY{n}{m}\PY{p}{,}\PY{n}{b}\PY{p}{)}\PY{p}{:}
             \PY{k}{return}  \PY{n+nb}{sum}\PY{p}{(}\PY{p}{[}\PY{p}{(}\PY{n}{y}\PY{o}{\PYZhy{}}\PY{n}{m}\PY{o}{*}\PY{n}{x} \PY{o}{\PYZhy{}}\PY{n}{b}\PY{p}{)}\PY{o}{*}\PY{o}{*}\PY{l+m+mi}{2} \PY{k}{for} \PY{n}{x}\PY{p}{,}\PY{n}{y} \PY{o+ow}{in} \PY{n+nb}{zip}\PY{p}{(}\PY{n}{X}\PY{p}{,}\PY{n}{Y}\PY{p}{)}\PY{p}{]}\PY{p}{)}\PY{o}{/}\PY{n+nb}{len}\PY{p}{(}\PY{n}{X}\PY{p}{)} 
         
         \PY{c+c1}{\PYZsh{} TODO 检查这里的结果, 如果你上面猜测的直线准确, 这里的输出会在1.5以内}
         \PY{n+nb}{print}\PY{p}{(}\PY{n}{calculateMSE2D}\PY{p}{(}\PY{n}{X}\PY{p}{,}\PY{n}{Y}\PY{p}{,}\PY{n}{m1}\PY{p}{,}\PY{n}{b1}\PY{p}{)}\PY{p}{)}
\end{Verbatim}


    \begin{Verbatim}[commandchars=\\\{\}]
1.2004601832217412

    \end{Verbatim}

    \subsubsection{\texorpdfstring{3.2.3 调整参数 \(m, b\)
来获得最小的平方平均误差}{3.2.3 调整参数 m, b 来获得最小的平方平均误差}}\label{ux8c03ux6574ux53c2ux6570-m-b-ux6765ux83b7ux5f97ux6700ux5c0fux7684ux5e73ux65b9ux5e73ux5747ux8befux5dee}

你可以调整3.2.1中的参数 \(m1,b1\) 让蓝点均匀覆盖在红线周围,然后微调
\(m1, b1\) 让MSE最小。

    \subsection{\texorpdfstring{3.3 (选做) 找到参数 \(m, b\)
使得平方平均误差最小}{3.3 (选做) 找到参数 m, b 使得平方平均误差最小}}\label{ux9009ux505a-ux627eux5230ux53c2ux6570-m-b-ux4f7fux5f97ux5e73ux65b9ux5e73ux5747ux8befux5deeux6700ux5c0f}

\textbf{这一部分需要简单的微积分知识( \$ (x\^{}2)' = 2x \$
)。因为这是一个线性代数项目,所以设为选做。}

刚刚我们手动调节参数,尝试找到最小的平方平均误差。下面我们要精确得求解
\(m, b\) 使得平方平均误差最小。

定义目标函数 \(E\) 为 \[
E = \frac{1}{2}\sum_{i=1}^{n}{(y_i - mx_i - b)^2}
\]

因为 \(E = \frac{n}{2}MSE\), 所以 \(E\) 取到最小值时,\(MSE\)
也取到最小值。要找到 \(E\) 的最小值,即要找到 \(m, b\) 使得 \(E\) 相对于
\(m\), \(E\) 相对于 \(b\) 的偏导数等于0.

因此我们要解下面的方程组。

\[
\begin{cases}
\displaystyle
\frac{\partial E}{\partial m} =0 \\
\\
\displaystyle
\frac{\partial E}{\partial b} =0 \\
\end{cases}
\]

\subsubsection{3.3.1
计算目标函数相对于参数的导数}\label{ux8ba1ux7b97ux76eeux6807ux51fdux6570ux76f8ux5bf9ux4e8eux53c2ux6570ux7684ux5bfcux6570}

首先我们计算两个式子左边的值

证明/计算: \[
\frac{\partial E}{\partial m} = \sum_{i=1}^{n}{-x_i(y_i - mx_i - b)}
\]

\[
\frac{\partial E}{\partial b} = \sum_{i=1}^{n}{-(y_i - mx_i - b)}
\]

    TODO 证明:

    \subsubsection{3.3.2 实例推演}\label{ux5b9eux4f8bux63a8ux6f14}

现在我们有了一个二元二次方程组

\[
\begin{cases}
\displaystyle
\sum_{i=1}^{n}{-x_i(y_i - mx_i - b)} =0 \\
\\
\displaystyle
\sum_{i=1}^{n}{-(y_i - mx_i - b)} =0 \\
\end{cases}
\]

为了加强理解,我们用一个实际例子演练。

我们要用三个点 \((1,1), (2,2), (3,2)\) 来拟合一条直线 y = m*x + b,
请写出

\begin{itemize}
\tightlist
\item
  目标函数 \(E\),
\item
  二元二次方程组,
\item
  并求解最优参数 \(m, b\)
\end{itemize}

    TODO 写出目标函数,方程组和最优参数

    \subsubsection{3.3.3
将方程组写成矩阵形式}\label{ux5c06ux65b9ux7a0bux7ec4ux5199ux6210ux77e9ux9635ux5f62ux5f0f}

我们的二元二次方程组可以用更简洁的矩阵形式表达,将方程组写成矩阵形式更有利于我们使用
Gaussian Jordan 消元法求解。

请证明 \[
\begin{bmatrix}
    \frac{\partial E}{\partial m} \\
    \frac{\partial E}{\partial b} 
\end{bmatrix} = X^TXh - X^TY
\]

其中向量 \(Y\), 矩阵 \(X\) 和 向量 \(h\) 分别为 : \[
Y =  \begin{bmatrix}
    y_1 \\
    y_2 \\
    ... \\
    y_n
\end{bmatrix}
,
X =  \begin{bmatrix}
    x_1 & 1 \\
    x_2 & 1\\
    ... & ...\\
    x_n & 1 \\
\end{bmatrix},
h =  \begin{bmatrix}
    m \\
    b \\
\end{bmatrix}
\]

    TODO 证明:

    至此我们知道,通过求解方程 \(X^TXh = X^TY\)
来找到最优参数。这个方程十分重要,他有一个名字叫做 \textbf{Normal
Equation},也有直观的几何意义。你可以在
\href{http://open.163.com/movie/2010/11/J/U/M6V0BQC4M_M6V2AJLJU.html}{子空间投影}
和
\href{http://open.163.com/movie/2010/11/P/U/M6V0BQC4M_M6V2AOJPU.html}{投影矩阵与最小二乘}
看到更多关于这个方程的内容。

    \subsubsection{\texorpdfstring{3.4 求解
\(X^TXh = X^TY\)}{3.4 求解 X\^{}TXh = X\^{}TY}}\label{ux6c42ux89e3-xtxh-xty}

在3.3 中,我们知道线性回归问题等价于求解 \(X^TXh = X^TY\)
(如果你选择不做3.3,就勇敢的相信吧,哈哈)

    \begin{Verbatim}[commandchars=\\\{\}]
{\color{incolor}In [{\color{incolor}27}]:} \PY{c+c1}{\PYZsh{} TODO 实现线性回归}
         \PY{l+s+sd}{\PYZsq{}\PYZsq{}\PYZsq{}}
         \PY{l+s+sd}{参数:X, Y 存储着一一对应的横坐标与纵坐标的两个一维数组}
         \PY{l+s+sd}{返回:线性回归的系数(如上面所说的 m, b)}
         \PY{l+s+sd}{\PYZsq{}\PYZsq{}\PYZsq{}}
         \PY{k}{def} \PY{n+nf}{linearRegression2D}\PY{p}{(}\PY{n}{X}\PY{p}{,}\PY{n}{Y}\PY{p}{)}\PY{p}{:} 
             \PY{c+c1}{\PYZsh{} 一维数组X转矩阵}
             \PY{n}{matxX} \PY{o}{=} \PY{p}{[}\PY{p}{[}\PY{n}{x}\PY{p}{,} \PY{l+m+mi}{1}\PY{p}{]} \PY{k}{for} \PY{n}{x} \PY{o+ow}{in} \PY{n}{X}\PY{p}{]}
             \PY{c+c1}{\PYZsh{} 矩阵转置}
             \PY{n}{transX} \PY{o}{=} \PY{n}{transpose}\PY{p}{(}\PY{n}{matxX}\PY{p}{)}
             \PY{c+c1}{\PYZsh{} 矩阵相乘}
             \PY{n}{matxA} \PY{o}{=} \PY{n}{matxMultiply}\PY{p}{(}\PY{n}{transX}\PY{p}{,} \PY{n}{matxX}\PY{p}{)}
         
             \PY{c+c1}{\PYZsh{} 一维数组Y转矩阵}
             \PY{n}{matxY} \PY{o}{=} \PY{p}{[}\PY{p}{[}\PY{n}{y}\PY{p}{]} \PY{k}{for} \PY{n}{y} \PY{o+ow}{in} \PY{n}{Y}\PY{p}{]}
             \PY{c+c1}{\PYZsh{} 矩阵相乘}
             \PY{n}{matxb} \PY{o}{=} \PY{n}{matxMultiply}\PY{p}{(}\PY{n}{transX}\PY{p}{,} \PY{n}{matxY}\PY{p}{)}
         
             \PY{c+c1}{\PYZsh{} 高斯消元求解}
             \PY{n}{gj\PYZus{}result} \PY{o}{=} \PY{n}{gj\PYZus{}Solve}\PY{p}{(}\PY{n}{matxA}\PY{p}{,} \PY{n}{matxb}\PY{p}{)}
         
         \PY{c+c1}{\PYZsh{}     print(gj\PYZus{}result)}
         
             \PY{n}{m}\PY{p}{,} \PY{n}{b} \PY{o}{=} \PY{l+m+mf}{0.0}\PY{p}{,} \PY{l+m+mf}{0.0}
         
             \PY{k}{if} \PY{o+ow}{not} \PY{n}{gj\PYZus{}result}\PY{p}{:}
                 \PY{k}{return} \PY{n}{m}\PY{p}{,} \PY{n}{b}
         
             \PY{n}{gj\PYZus{}res\PYZus{}len} \PY{o}{=} \PY{n+nb}{len}\PY{p}{(}\PY{n}{gj\PYZus{}result}\PY{p}{)}
             \PY{k}{if} \PY{n}{gj\PYZus{}res\PYZus{}len} \PY{o}{\PYZgt{}} \PY{l+m+mi}{0}\PY{p}{:}
                 \PY{n}{m} \PY{o}{=} \PY{n}{gj\PYZus{}result}\PY{p}{[}\PY{l+m+mi}{0}\PY{p}{]}\PY{p}{[}\PY{l+m+mi}{0}\PY{p}{]}
             \PY{k}{if} \PY{n}{gj\PYZus{}res\PYZus{}len} \PY{o}{\PYZgt{}} \PY{l+m+mi}{1}\PY{p}{:}
                 \PY{n}{b} \PY{o}{=} \PY{n}{gj\PYZus{}result}\PY{p}{[}\PY{l+m+mi}{1}\PY{p}{]}\PY{p}{[}\PY{l+m+mi}{0}\PY{p}{]}
             \PY{k}{return} \PY{n}{m}\PY{p}{,} \PY{n}{b}
         
         \PY{n}{m2}\PY{p}{,}\PY{n}{b2} \PY{o}{=} \PY{n}{linearRegression2D}\PY{p}{(}\PY{n}{X}\PY{p}{,}\PY{n}{Y}\PY{p}{)}
         \PY{k}{assert} \PY{n+nb}{isinstance}\PY{p}{(}\PY{n}{m2}\PY{p}{,}\PY{n+nb}{float}\PY{p}{)}\PY{p}{,}\PY{l+s+s2}{\PYZdq{}}\PY{l+s+s2}{m is not a float}\PY{l+s+s2}{\PYZdq{}}
         \PY{k}{assert} \PY{n+nb}{isinstance}\PY{p}{(}\PY{n}{b2}\PY{p}{,}\PY{n+nb}{float}\PY{p}{)}\PY{p}{,}\PY{l+s+s2}{\PYZdq{}}\PY{l+s+s2}{b is not a float}\PY{l+s+s2}{\PYZdq{}}
         \PY{n+nb}{print}\PY{p}{(}\PY{n}{m2}\PY{p}{,}\PY{n}{b2}\PY{p}{)}
\end{Verbatim}


    \begin{Verbatim}[commandchars=\\\{\}]
3.084 12.6578

    \end{Verbatim}

    \begin{Verbatim}[commandchars=\\\{\}]
{\color{incolor}In [{\color{incolor}28}]:} \PY{c+c1}{\PYZsh{} 请不要修改下面的代码}
         \PY{n}{m2}\PY{p}{,}\PY{n}{b2} \PY{o}{=} \PY{n}{linearRegression2D}\PY{p}{(}\PY{n}{X}\PY{p}{,}\PY{n}{Y}\PY{p}{)}
         \PY{k}{assert} \PY{n+nb}{isinstance}\PY{p}{(}\PY{n}{m2}\PY{p}{,}\PY{n+nb}{float}\PY{p}{)}\PY{p}{,}\PY{l+s+s2}{\PYZdq{}}\PY{l+s+s2}{m is not a float}\PY{l+s+s2}{\PYZdq{}}
         \PY{k}{assert} \PY{n+nb}{isinstance}\PY{p}{(}\PY{n}{b2}\PY{p}{,}\PY{n+nb}{float}\PY{p}{)}\PY{p}{,}\PY{l+s+s2}{\PYZdq{}}\PY{l+s+s2}{b is not a float}\PY{l+s+s2}{\PYZdq{}}
         \PY{n+nb}{print}\PY{p}{(}\PY{n}{m2}\PY{p}{,}\PY{n}{b2}\PY{p}{)}
\end{Verbatim}


    \begin{Verbatim}[commandchars=\\\{\}]
3.084 12.6578

    \end{Verbatim}

    你求得的回归结果是什么? 请使用运行以下代码将它画出来。

    \begin{Verbatim}[commandchars=\\\{\}]
{\color{incolor}In [{\color{incolor}29}]:} \PY{c+c1}{\PYZsh{}\PYZsh{} 请不要修改下面的代码}
         \PY{n}{vs\PYZus{}scatter\PYZus{}2d}\PY{p}{(}\PY{n}{X}\PY{p}{,} \PY{n}{Y}\PY{p}{,} \PY{n}{m2}\PY{p}{,} \PY{n}{b2}\PY{p}{)}
         \PY{n+nb}{print}\PY{p}{(}\PY{n}{calculateMSE2D}\PY{p}{(}\PY{n}{X}\PY{p}{,}\PY{n}{Y}\PY{p}{,}\PY{n}{m2}\PY{p}{,}\PY{n}{b2}\PY{p}{)}\PY{p}{)}
\end{Verbatim}


    
    \begin{verbatim}
<IPython.core.display.Javascript object>
    \end{verbatim}

    
    
    \begin{verbatim}
<IPython.core.display.HTML object>
    \end{verbatim}

    
    \begin{Verbatim}[commandchars=\\\{\}]
1.0209278908110677

    \end{Verbatim}

    \subsection{Bonus !!!}\label{bonus}

如果你的高斯约当消元法通过了单元测试, 那么它将能够解决多维的回归问题\\
你将会在更高维度考验你的线性回归实现

    \begin{Verbatim}[commandchars=\\\{\}]
{\color{incolor}In [{\color{incolor}30}]:} \PY{c+c1}{\PYZsh{} 生成三维的数据点}
         \PY{n}{X\PYZus{}3d}\PY{p}{,} \PY{n}{Y\PYZus{}3d} \PY{o}{=} \PY{n}{generatePoints3D}\PY{p}{(}\PY{n}{seed}\PY{p}{)}
         \PY{n}{vs\PYZus{}scatter\PYZus{}3d}\PY{p}{(}\PY{n}{X\PYZus{}3d}\PY{p}{,} \PY{n}{Y\PYZus{}3d}\PY{p}{)}
\end{Verbatim}


    
    \begin{verbatim}
<IPython.core.display.Javascript object>
    \end{verbatim}

    
    
    \begin{verbatim}
<IPython.core.display.HTML object>
    \end{verbatim}

    
    你的线性回归是否能够对付三维的情况?

    \begin{Verbatim}[commandchars=\\\{\}]
{\color{incolor}In [{\color{incolor}31}]:} \PY{k}{def} \PY{n+nf}{linearRegression}\PY{p}{(}\PY{n}{X}\PY{p}{,}\PY{n}{Y}\PY{p}{)}\PY{p}{:}
             \PY{k}{return} \PY{k+kc}{None}
\end{Verbatim}


    \begin{Verbatim}[commandchars=\\\{\}]
{\color{incolor}In [{\color{incolor}32}]:} \PY{n}{coeff} \PY{o}{=} \PY{n}{linearRegression}\PY{p}{(}\PY{n}{X\PYZus{}3d}\PY{p}{,} \PY{n}{Y\PYZus{}3d}\PY{p}{)}
         \PY{n}{vs\PYZus{}scatter\PYZus{}3d}\PY{p}{(}\PY{n}{X\PYZus{}3d}\PY{p}{,} \PY{n}{Y\PYZus{}3d}\PY{p}{,} \PY{n}{coeff}\PY{p}{)}
\end{Verbatim}


    
    \begin{verbatim}
<IPython.core.display.Javascript object>
    \end{verbatim}

    
    
    \begin{verbatim}
<IPython.core.display.HTML object>
    \end{verbatim}

    

    % Add a bibliography block to the postdoc
    
    
    
    \end{document}
